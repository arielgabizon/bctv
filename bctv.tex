\def\notes{1}
\documentclass[11pt]{article}
%\usepackage[T1]{fontenc}
%\usepackage[letterpaper]{geometry}
%\geometry{verbose,tmargin=3cm,bmargin=3cm,lmargin=3cm,rmargin=3cm}
%\usepackage{verbatim}
\usepackage{amsfonts,amsmath,amssymb,amsthm}
%\usepackage{setspace}
\usepackage{xspace}%,enumitem}
%\usepackage{times}
\usepackage{fullpage}
%\usepackage{hyperref}
%\doublespacing
\usepackage{color}
\usepackage{numdef}
\usepackage{enumitem}
\usepackage{amsopn}
%\usepackage{hyperref} 

\providecommand{\sqbinom}{\genfrac{\lbrack}{\rbrack}{0pt}{}}
\DeclareMathOperator{\spn}{span}

\newif\ifdraft
%\drafttrue
\draftfalse
%%%%%%%%%%%%%%%%%%%%%%%%%%%%% Textclass specific LaTeX commands.
%\numberwithin{equation}{section} %% Comment out for sequentially-numbered
\numberwithin{figure}{section} %% Comment out for sequentially-numbered
\newtheorem{thm}{Theorem}[section]
\newtheorem{conjecture}[thm]{Conjecture}
\newtheorem{definition}[thm]{Definition}
\newtheorem{dfn}[thm]{Definition}
\newtheorem{lemma}[thm]{Lemma}
\newtheorem{remark}[thm]{Remark}
\newtheorem{proposition}[thm]{Proposition}
\newtheorem{corollary}[thm]{Corollary}
\newtheorem{claim}[thm]{Claim}
\newtheorem{fact}[thm]{Fact}
\newtheorem{openprob}[thm]{Open Problem}
\newtheorem{remk}[thm]{Remark}
\newtheorem{example}[thm]{Example}
\newtheorem{apdxlemma}{Lemma}
\newcommand{\question}[1]{{\sf [#1]\marginpar{?}} }
%\usepackage{babel}
\usepackage{tikz}

\makeatletter
\newcommand*{\circled}{\@ifstar\circledstar\circlednostar}
\makeatother

\newcommand*\circledstar[1]{%
  \tikz[baseline=(C.base)]
    \node[%
      fill,
      circle,
      minimum size=1.35em,
      text=white,
      font=\sffamily,
      inner sep=0.5pt
    ](C) {#1};%
}
\newcommand*\circlednostar[1]{%
  \tikz[baseline=(C.base)]
    \node[%
      draw,
      circle,
      minimum size=1.35em,
      font=\sffamily,
      inner sep=0.5pt
    ](C) {#1};%
}

\providecommand{\bd}[1]{\circled{#1}}


%Eli's macros
\newcommand{\E}{{\mathbb{E}}}
%\newcommand{\N}{{\mathbb{N}}}
\newcommand{\R}{{\mathbb{R}}}
\newcommand{\calF}{{\cal{F}}}
%\newcommand{\inp}[1]{\langle{#1}\rangle}
\newcommand{\disp}[1]{D_{#1}}
%\newcommand{\spn}[1]{{\rm{span}}\left(#1\right)}
%\newcommand{\supp}{{\rm{supp}}}
\newcommand{\agree}{{\sf{agree}}}
\newcommand{\RS}{{\sf{RS}}}
\newcommand{\aRS}{{\sf{aRS}}}
\newcommand{\rrs}{R_{\aRS}}
\newcommand{\VRS}{V_\RS}
\newcommand{\set}[1]{\ensuremath{\left\{#1\right\}}\xspace}
\newcommand{\angles}[1]{\langle{#1}\rangle}
\newcommand{\condset}[2]{\set{#1 \mid #2 }}
\newcommand{\zo}{\set{0,1}}
\newcommand{\zon}{{\zo^n}}
\newcommand{\zom}{{\zo^m}}
\newcommand{\zok}{{\zo^}}
\newcommand{\eps}{\epsilon}
\newcommand{\e}{\eps}
%\newcommand{\d}{\delta}
\newcommand{\poly}{{\rm{poly}}}
\newcommand{\itone}{{\it{(i)}}\xspace}
\newcommand{\ittwo}{{\it{(ii)}}\xspace}
\newcommand{\itthree}{{\it{(iii)}}\xspace}
\newcommand{\itfour}{{\it{(iv)}}\xspace}
\newcommand{\cc}{{\rm{CC}}}
\newcommand{\rnk}{{\rm{rank}}}
\newcommand{\T}{T}
\newcommand{\I}{{\mathbb{I}}}
\ifdraft
\newcommand{\ariel}[1]{{\color{blue}{\textit{#1 --- ariel gabizon}}}}
\else
\newcommand{\ariel}[1]{}
\fi
\providecommand{\improvement}[1]{{\color{red} \textbf{#1}}}
\title{%
On the security of the BCTV Pinocchio zk-SNARK Variant}
\date{\today}
\author{Ariel Gabizon}
\date{Zcash Company}
\begin{document}
\maketitle
 \mathchardef\mhyphen="2D

\num\newcommand{\G1}{\ensuremath{{\mathbb G}_1}\xspace}
\num\newcommand{\G2}{\ensuremath{{\mathbb G}_2}\xspace}
\num\newcommand{\G11}{\ensuremath{\G1\setminus \set{0} }\xspace}
\num\newcommand{\G21}{\ensuremath{\G2\setminus \set{0} }\xspace}
\newcommand{\grouppair}{\ensuremath{G^*}\xspace}

\newcommand{\Gt}{\ensuremath{{\mathbb G}_t}\xspace}
\newcommand{\F}{\ensuremath{\mathbb F}\xspace}
\newcommand{\help}[1]{$#1$-helper\xspace}
\newcommand{\randompair}[1]{\ensuremath{\mathsf{randomPair}(#1)}\xspace}
\newcommand{\pair}[1]{$#1$-pair\xspace}
\newcommand{\pairs}[1]{$#1$-pairs\xspace}

\newcommand{\pairone}[1]{\G1-$#1$-pair\xspace}
\newcommand{\pairtwo}[1]{\G2-$#1$-pair\xspace}
\newcommand{\sameratio}[2]{\ensuremath{\mathsf{SameRatio}(#1,#2)}\xspace}
\newcommand{\vecc}[2]{\ensuremath{(#1)_{#2}}\xspace}
\newcommand{\players}{\ensuremath{[n]}\xspace}
\newcommand{\adv}{\ensuremath{A}\xspace}
\newcommand{\ci}{\ensuremath{\mathrm{CI}}\xspace}
\newcommand{\pairvec}[1]{$#1$-vector\xspace}
\newcommand{\Fq}{\ensuremath{\mathbb{F}_q}\xspace}
\newcommand{\randpair}[1]{\ensuremath{\mathsf{rp}_{#1}}\xspace}
\newcommand{\randpairone}[1]{\ensuremath{\mathsf{rp}_{#1}^{1}}\xspace}

\newcommand{\randpairtwo}[1]{\ensuremath{\mathsf{rp_{#1}^2}}\xspace}%the randpair in G2

\newcommand{\rej}{\ensuremath{\mathsf{rej}}\xspace}
\newcommand{\acc}{\ensuremath{\mathsf{acc}}\xspace}
\newcommand{\sha}[1]{\ensuremath{\mathsf{COMMIT}(#1)}\xspace}
 \newcommand{\shaa}{\ensuremath{\mathsf{COMMIT}}\xspace}
 \newcommand{\comm}[1]{\ensuremath{\mathsf{comm}_{#1}}\xspace}
 \newcommand{\defeq}{:=}

\newcommand{\A}{\ensuremath{\vec{A}}\xspace}
\newcommand{\B}{\ensuremath{\vec{B}}\xspace}
\newcommand{\C}{\ensuremath{\vec{C}}\xspace}
\newcommand{\Btwo}{\ensuremath{\vec{B_2}}\xspace}
\newcommand{\treevecsimp}{\ensuremath{(\tau,\rho_A,\rho_A \rho_B,\rho_A\alpha_A,\rho_A\rho_B\alpha_B, \rho_A\rho_B\alpha_C,\beta,\beta\gamma)}\xspace}% The sets of elements used in simplifed relation tree in main text body
\newcommand{\rcptc}{random-coefficient subprotocol\xspace}
\newcommand{\rcptcparams}[2]{\ensuremath{\mathrm{RCPC}(#1,#2)}\xspace}
\newcommand{\verifyrcptcparams}[2]{\ensuremath{\mathrm{\mathsf{verify}RCPC}(#1,#2)}\xspace}

 \num\newcommand{\ex1}[1]{\ensuremath{ #1\cdot g_1}\xspace}
 \num\newcommand{\ex2}[1]{\ensuremath{#1\cdot g_2}\xspace}
 \newcommand{\pr}{\mathrm{Pr}}
 \newcommand{\powervec}[2]{\ensuremath{(1,#1,#1^{2},\ldots,#1^{#2})}\xspace}
\num\newcommand{\out1}[1]{\ensuremath{\ex1{\powervec{#1}{d}}}\xspace}
\num\newcommand{\out2}[1]{\ensuremath{\ex2{\powervec{#1}{d}}}\xspace}
 \newcommand{\nizk}[2]{\ensuremath{\mathrm{NIZK}(#1,#2)}\xspace}% #2 is the hash concatenation input
 \newcommand{\verifynizk}[3]{\ensuremath{\mathrm{VERIFY\mhyphen NIZK}(#1,#2,#3)}\xspace}
\newcommand{\protver}{protocol verifier\xspace} 
\newcommand{\mulgroup}{\ensuremath{\F^*}\xspace}
\newcommand{\lag}[1]{\ensuremath{L_{#1}}\xspace} 
\newcommand{\sett}[2]{\ensuremath{\set{#1}_{#2}}\xspace}
\newcommand{\omegaprod}{\ensuremath{\alpha_{\omega}}\xspace}
\newcommand{\lagvec}[1]{\ensuremath{\mathrm{LAG}_{#1}}\xspace}

\num\newcommand{\enc1}[1]{\ensuremath{\left[#1\right]_1}\xspace}
\num\newcommand{\enc2}[1]{\ensuremath{\left[#1\right]_2}\xspace}

\num\newcommand{\G0}{\ensuremath{\mathbf{G}}\xspace}
\newcommand{\GG}{\ensuremath{\mathbf{G^*}}\xspace}  % would have liked to call this G01 but problem with name
\num\newcommand{\g0}{\ensuremath{\mathbf{g}}\xspace}
\newcommand{\inst}{\ensuremath{\phi}\xspace}
\newcommand{\inp}{\ensuremath{x}\xspace}
\newcommand{\wit}{\ensuremath{\omega}\xspace}
\newcommand{\ver}{\ensuremath{\mathsf{V}}\xspace}
\newcommand{\per}{\ensuremath{\mathsf{P}}\xspace}

\newcommand{\prf}{\ensuremath{\pi}\xspace}
\newcommand{\ext}{\ensuremath{E}\xspace}
\newcommand{\params}{\ensuremath{\mathsf{params}_{\inst}}\xspace}
\newcommand{\protparams}{\ensuremath{\mathsf{params}_{\inst}^\advv}\xspace}
\num\newcommand{\p1}{\ensuremath{P_1}\xspace}
\newcommand{\advv}{\ensuremath{B}\xspace} % the adversary that uses protocol adversary as black box
\renewcommand{\sim}{\ensuremath{\mathsf{sim}}\xspace}%the distribution of messages when \advv simulates message of \p1
\newcommand{\real}{\ensuremath{\mathsf{real}}\xspace}%the distribution of messages when \p1 is honest and \adv controls rest of players
 \newcommand{\koevec}[2]{\ensuremath{(1,#1,\ldots,#1^{#2},\alpha,\alpha #1,\ldots,\alpha #1^{#2})}\xspace}
\newcommand{\mida}{\ensuremath{A_{\mathrm{mid}}}\xspace}
\newcommand{\midb}{\ensuremath{B_{\mathrm{mid}}}\xspace}
\newcommand{\midc}{\ensuremath{C_{\mathrm{mid}}}\xspace}
\newcommand{\chal}{\ensuremath{\mathsf{challenge}}\xspace}
\newcommand{\attackparams}{\ensuremath{\mathsf{params^{pin}}}\xspace}
\newcommand{\pk}{\ensuremath{\mathsf{pk}}\xspace}
\newcommand{\attackdist}[2]{\ensuremath{AD_{#1}}\xspace}
\renewcommand{\neg}{\ensuremath{\mathsf{negl}(\log r)}\xspace}
\newcommand{\ro}{\ensuremath{{\mathcal R}}\xspace}
\newcommand{\elements}[1]{\ensuremath{\mathsf{elements}_{#1}}\xspace}
 \num\newcommand{\elmpowers1}[1]{\ensuremath{\mathrm{\mathsf{e}}^1_{#1}}\xspace}
 \num\newcommand{\elmpowers2}[1]{\ensuremath{\mathrm{\mathsf{e}}^2_{#1}}\xspace}
\newcommand{\elempowrs}[1]{\ensuremath{\mathsf{e}_{#1}}\xspace}
 \newcommand{\secrets}{\ensuremath{\mathsf{secrets}}\xspace}
 \newcommand{\polysofdeg}[1]{\ensuremath{\F_{< #1}[X]}\xspace}
\newcommand{\bctv}{\ensuremath{\mathsf{BCTV}}\xspace}
\newcommand{\PI}{\ensuremath{\mathsf{PI}}\xspace}
\newcommand{\dl}[1]{\ensuremath{\widehat{#1}}\xspace}
\begin{abstract}

The main result of this note is a severe flaw in the description of the zk-SNARK in \cite{BCTV}.
The flaw stems from including redundant elements in the CRS, as compared to that of the original Pinocchio protocol \cite{PGHR}, which are vital not to expose.
The flaw enables creating a proof of knowledge for \emph{any} public input given a valid proof for \emph{some} public input.
We also provide a proof of security for the \cite{BCTV} zk-SNARK in the generic group model, when these elements are excluded from the CRS, provided a certain linear algebraic condition is satisfied by the QAP polynomials.
\end{abstract}

\section{Introduction}

Parno et. al \cite{PGHR} presented a zk-SNARK construction based on the breakthrough work of \cite{GGPR} that they called Pinocchio.
Ben-Sasson et. al \cite{BCTV} presented a variant of Pinocchio with the advantage of shorter verification time and verification key length, at the expense of an arguably negligible increase in proving time. However, \cite{BCTV} did not present a security proof for this variant, and in fact Parno \cite{Parno15} found an attack against the \cite{BCTV} SNARK and suggested to mitigate it by imposing a certain linear independence condition on some of the public instance polynomials -- which \cite{BCTV} did in a revised version of the paper and corresponding implementation \cite{libsnark}. In this note, we show a more severe attack on \cite{BCTV} that takes advantage of redundant elements in the proving key that should have been omitted.


\subsection{Impacted work}
The first proof of security for \cite{BCTV} was given in \cite{BGG}. However, the proof has an error and in fact we discovered the attack while reviewing it.
Other papers that leverage the \cite{BCTV} construction inherit the flaw also; the ones we found are \cite{Adsnark,Fuchsbauer18}. The flaw also appears in Parno's \cite{Parno15} analysis of \cite{BCTV} which had identified a separate knowledge soundness flaw in the construction.

The (\texttt{libsnark} ~\cite{libsnark}) implementation of the \cite{BCTV} prover distributed alongside the paper expects that the elements of concern are removed from the CRS, apparently intended as one of several optimizations that deviated from the construction as described in the paper. (The elements are not used by the prover.) As a result, parameters constructed by \texttt{libsnark} are not vulnerable to this specific attack.

However, the multi-party computation \cite{BGG} follows the construction described in \cite{BCTV} closely and so the elements are produced during the protocol's execution, though they do not appear in the final parameters in order to be compatible with the \texttt{libsnark} prover. Unfortunately, the transcript of the protocol's execution must be distributed so that the resulting parameters can be verified, and so the elements are revealed.

We also discovered an independent implementation of \cite{BCTV} which appears to inherit the flaw~\cite{snarkjs}, as unlike \texttt{libsnark} its key generation and proving routines match the paper closely.

\subsection{Notation}
We will be working over bilinear groups $\G1$, $\G2$, and $\Gt$ each of prime order $p$, together with respective generators $g_1$, $g_2$ and $g_T$.
These groups are equipped with a non-degenerate bilinear pairing $e:\G1 \times \G2 \to \Gt$, with $e(g_1, g_2) = g_T$.
We write $\G1$ and $\G2$ additively, and $\Gt$ multiplicatively.
We denote by \F the field of the same order $p$.
For $a \in \F$, we denote $\enc1{a}\defeq a\cdot g_1, \enc2{a}\defeq a\cdot g_2$.
% We use the notation $\G0\defeq \G1\times \G2$ and $\g0\defeq (g_1,g_2)$.
% Given an element $h \in \G0$, we denote by $h_1 (h_2)$ the $\G1 (\G2)$ element of $h$.
% We denote by $\G11,\G21$ the non-zero elements of $\G1,\G2$ and denote $\GG \defeq \G11\times \G21$.

\subsection{Description of \cite{BCTV} SNARK}\label{subsec:bctv}

We recall the zk-SNARK of \cite{BCTV} as described in that paper.
We assume familiary of quadratic arithmetic programs.
See e.g., Section 2.3 in \cite{groth16} for definitions.
We use similar notation to \cite{BCTV}, denoting by $m$ the size of the QAP, $d$ the degree and $n$ the number of public inputs.
More specifically, our QAP has the form \set{\sett{A_i(X),B_i(X),C_i(X)}{i\in [0..m]}, Z(X)}
where $A_i,B_i,C_i\in \F[X]$ have degree at most\footnote{\cite{BCTV} define $A_i,B_i,C_i$ to be of degree strictly less than $d$, however since $Z$ needs to be later added to \set{A_i},\set{B_i},\set{C_i} for zero-knowledge, it is more convenient for us to allow degree at most $d$ and assume $Z$ is already included. Note that in terms of the set of satisfiable instances $x\in \F^n$ every degree $d$ QAP is equivalent to one where \set{A_i,B_i,C_i} have degree smaller than $d$ obtained by taking the original polynomials mod $Z$.} $d$, and $Z\in \F[X]$ has degree exactly $d$.


We proceed to describe the proving system of \cite{BCTV}.
We assume we are already given a description of the groups $\G1,\G2,\Gt$, the pairing $e$, and uniformly chosen generators 
$g_1\in \G1,g_2\in \G2$, and these are all public.
\paragraph{\underline{\bctv key generation:}}

\begin{enumerate}
 \item Sample random $\tau,\rho_A,\rho_B,\alpha_A,\alpha_B,\alpha_C,\gamma,\beta\in \F^*$
 \item For $i\in [0..d]$ output $\pk_{H,i}\defeq \enc1{\tau^i}$
 \item For $i\in [0..m]$ output
 
 \begin{enumerate}
  \item $\pk_{A,i}\defeq \enc1{\rho_A A_i(\tau)}$
  
\item  $\pk'_{A,i}\defeq\enc1{\alpha_A\rho_A A_i(\tau)}$,
\item $\pk_{B,i}\defeq\enc2{\rho_B B_i(\tau)}$,
\item $\pk'_{B,i}\defeq \enc1{\alpha_B\rho_B B_i(\tau)}$,
\item $\pk_{C,i}\defeq \enc1{\rho_A\rho_B C_i(\tau)}$,
\item $\pk'_{C,i}\defeq \enc1{\alpha_C \rho_A\rho_B C_i(\tau)}$ 
\item $\pk_{K,i}\defeq \enc1{\beta (\rho_A A_i(\tau) + \rho_B B_i(\tau) + \rho_A\rho_B C_i(\tau))}$ 

\end{enumerate}
\item Output the additional verification key elements $(\enc2{\alpha_A},\enc1{\alpha_B},\enc2{\alpha_C},\enc2{\gamma},\enc1{\beta\gamma},\enc2{\beta\gamma},\enc2{\rho_A\rho_B Z(\tau)})$
 \end{enumerate}

\newcommand{\Amid}{\ensuremath{A_{\mathrm{mid}}}\xspace}
\paragraph{\underline{\bctv prover:}\\}
The prover has in his hand a QAP solution $(x_0=1,x_1,\ldots,x_m)$ that coincides with the public input $x=(x_1,\ldots,x_n)$ and satisfies the following:
If we define $A\defeq \sum_{i=0}^m x_i\cdot A_i, B\defeq \sum_{i=0}^m x_i \cdot B_i,$ and $C\defeq \sum_{i=0}^m x_i \cdot C_i$;
then the polynomial $P\defeq A\cdot B - C$ will be divisble by the target polynomial $Z$, and \per can compute the polynomial $H$ of degree at most $d$ with $P=H\cdot Z$. Let $\Amid\defeq A -\sum_{i=0}^n x_i\cdot A_i$.

Given the proving key, \per computes as linear combinations of the proving key elements
\begin{enumerate}
\item $\pi_A\defeq \enc1{\rho_A \Amid(\tau)}$, $\pi'_A\defeq \enc1{\alpha_A\rho_A \Amid(\tau)}$.
\item $\pi_B\defeq \enc2{\rho_B B(\tau)}$, $\pi'_B \defeq \enc1{\alpha_B\rho_B B(\tau)}$.
\item $\pi_C\defeq \enc1{\rho_A\rho_B C(\tau)}$, $\pi'_C \defeq \enc1{\alpha_C\rho_A\rho_B C(\tau)}$.
\item $\pi_K\defeq \enc1{\beta(\rho_A A(\tau) + \rho_B B(\tau) + \rho_A\rho_B C(\tau))}$.
\item $\pi_H\defeq \enc1{H(\tau}$.
 \end{enumerate}
 and outputs $\prf = (\pi_A,\pi_B,\pi_C,\pi'_A,\pi'_B,\pi'_C,\pi_H,\pi_K)$,
 

\paragraph{\underline{\bctv verifier:}\\}
Denote the ``public input component'' 
\[ \PI(x) \defeq \pk_{A,0} + \sum_{i=1}^n x_i \pk_{A,i} = \enc1{\rho_A A_0(\tau) + \sum_{i=1}^n x_i \rho_A A_i(\tau) }\]
%  For an element $a$ of \G1 (\G2) denote by \dl{a} the discrete log of $a$ w.r.t $g_1(g_2)$.
 The verifier, using pairings and the verification key, checks the following.
\begin{enumerate}
 \item $e(\pi'_A,g_2)=e(\pi_A,\enc2{\alpha_A})$.
\item $e(\pi'_B, g_2)=e(\enc1{\alpha_B},\pi_B)$.
\item $e(\pi'_C,g_2)=e(\pi_C,\enc2{\alpha_C})$.
\item $e(\pi_K,\enc2{\gamma})=e(\PI(x)+\pi_A+\pi_C,\enc2{\beta\gamma})\cdot e(\enc1{\beta\gamma},\pi_B)$.
\item $e(\PI(x)+\pi_A,\pi_B)= e(\pi_C,g_2)\cdot e(\pi_H,\enc2{Z(\tau)\rho_A\rho_B})$.
 \end{enumerate}
 
 \section{Attack Description}\label{sec:attack}
 Note that the elements \sett{\pk'_{A,i}}{i\in [0..n]} are not used at all by the verifier and honest prover, and thus could have been omitted from the key. We show that these elements allow a malicious prover to replace the public input arbitrarily when starting from a valid proof.
 Loosely speaking, we do this by adding a factor to $\pi_A$ that ``switches'' the public input the proof is arguing about.
 The first verifier check -- the ``knowledge check'' for $\pi_A$ -- should catch this, but the redundant elements allow the malicious prover to add
 the analogous factor to $\pi'_A$ and pass the check. Details follow.
 
 
 Suppose we are given a valid proof $\prf = (\pi_A,\pi_B,\pi_C,\pi'_A,\pi'_B,\pi'_C,\pi_H,\pi_K)$
 for a public input $x=(x_1,\ldots,x_n)\in \F^{n}$.
 Choose any $x'=(x'_1,\ldots,x'_n)\in \F^n$.

 Set 
 \[\eta_A\defeq \pi_A + \sum_{i=1}^n (x_i-x'_i)\pk_{A,i}\]
 \[\eta'_A\defeq \pi'_A + \sum_{i=1}^n (x_i-x'_i) \pk'_{A,i}\]
 
 We claim that $\prf^* \defeq (\eta_A,\pi_B,\pi_C,\eta'_A,\pi'_B,\pi'_C,\pi_K,\pi_H)$ is a valid proof for
 public input $x'$.
 
 The verifier checks with public input $x'$ and proof $\prf^*$ are
\begin{enumerate}
 \item $e(\eta'_A,g_2)=e(\eta_A,\enc2{\alpha_A})$.
\item $e(\pi'_B, g_2)=e(\enc1{\alpha_B},\pi_B)$.
\item $e(\pi'_C,g_2)=e(\pi_C,\enc2{\alpha_C})$.
\item $e(\pi_K,\enc2{\gamma})=e(\PI(x')+\eta_A+\pi_C,\enc2{\beta\gamma})\cdot e(\enc1{\beta\gamma},\pi_B)$.
\item $e(\PI(x')+\eta_A,\pi_B)= e(\pi_C,g_2)\cdot e(\pi_H,\enc2{Z(\tau)\rho_A\rho_B})$.
 \end{enumerate}
 
 
 We show that the five equations all hold.
 \begin{enumerate}
  \item The check  $e(\eta'_A,g_2)=e(\eta_A,\enc2{\alpha_A})$;
  this is where the redundant elements crucially come into play.
  
  
  We have
  \[\eta'_A = \pi'_A + \sum_{i=1}^n (x_i-x'_i) \pk'_{A,i}\]
 So
 \[e(\eta'_A,g_2) = \e(\pi'_A,g_2)\cdot e\left(\sum_{i=1}^n (x_i-x'_i) \pk'_{A,i},g_2\right)\]
 Since \prf is a valid proof, this is:
  \[=e(\pi_A,\enc2{\alpha_A})\cdot e\left(\sum_{i=1}^n (x_i-x'_i) \pk'_{A,i},g_2\right)\]
  Using $\pk'_{A,i} = \alpha_A\cdot \pk_{A,i}$ for every $i$,
  \[=e(\pi_A,\enc2{\alpha_A})\cdot e\left(\alpha_A\cdot \left(\sum_{i=1}^n (x_i-x'_i) \pk_{A,i}\right),g_2\right)\]
Using bi-linearity of the pairing:
 \[=e(\pi_A,\enc2{\alpha_A})\cdot e\left(\sum_{i=1}^n (x_i-x'_i) \pk_{A,i},\enc2{\alpha_A}\right)\]
  \[=e\left(\pi_A+\sum_{i=1}^n (x_i-x'_i) \pk_{A,i},\enc2{\alpha_A}\right) = e(\eta_A,\enc2{\alpha_A}).\]
  \item The second and third checks involve the unchanged $\pi_B,\pi'_B,\pi_C,\pi'_C$
  and thus pass since \prf was valid.
  \item The fourth and fifth equations are also identical in \prf and $\prf^*$.
  The only difference is that the later replaces the term $\PI(x) + \pi_A$ 
  with $\PI(x')+ \eta_A$. And 
    \[\PI(x) + \pi_A = \pk_{A,0} + \sum_{i=1}^n x_i\pk_{A,i} + \pi_A = 
  \pk_{A,0} + \sum_{i=1}^n x'_i\pk_{A,i} +  \sum_{i=1}^n (x_i-x'_i) \pk_{A,i}  + \pi_A =
  \PI(x') + \eta_A.
  \]

 \end{enumerate}
 
 \renewcommand{\span}[1]{\ensuremath{\mathrm{Span}_{\F}(#1)}\xspace}
 \newcommand{\bctvprime}{\ensuremath{\bctv'}\xspace}
\section{Security Proof in the Generic Group Model}\label{sec:ggmproof}
Let us denote by \bctvprime the scheme identical to the one in \cite{BCTV} described in Subsection \ref{subsec:bctv},
with two modifications:
\begin{itemize}
 \item The elements $\sett{\pk_{A',i}}{i\in [0..n]}$ are excluded from the proving key.
 \item The scheme is only defined for QAPs where the polynomials \set{A_i} satisfy
   \begin{enumerate}
    \item The polynomials \sett{A_i}{i\in [0..n]} are linearly independent (this condition already appears in \cite{BCTV} at \cite{Parno15}'s suggestion).
    \item $\span{\sett{A_i}{i\in[0..n]}}\cap \span{\sett{A_i}{i\in [n+1..m]}}=\set{0}$. \footnote{In conversation with Alessandro Chiesa and Madars Virza, we  learned that they were aware of the necessity of this condition, and it is satisfied by any QAP constructed in libsnark\cite{libsnark}. It also already appears in \cite{BGG}.}
   \end{enumerate}
 \item We also assume, mainly as a convenience, that $Z\in \sett{C_i}{i\in [n+1..m]}$. This is always the case if we want zero-knowledge, and the \bctv construction adds $Z$ to $\set{A_i},\set{B_i},\set{C_i}$.
\end{itemize}

We show \bctvprime is knowledge sound in the generic group model.
We will use the following simple linear algebra claim.
\begin{claim}\label{clm:independence}
 Fix positive integers $n<m$. Let $v_0,\ldots,v_m$ be vectors in a vector space over \F such that
 \begin{enumerate}
  \item $v_0,\ldots,v_n$ are linearly independent.
  \item Defining $V = \span{v_0,\ldots,v_n}$ and $U=\span{v_{n+1},\ldots,v_m}$, we have $V\cap U =\set{0}$.
 \end{enumerate}
Suppose we are given $(x_0,\ldots,x_m),(a_0,\ldots,a_m)\in \F^m$ such that $\sum_{i=0}^m x_i\cdot v_i = \sum_{i=0}^m a_i\cdot v_i$.
Then $(x_0,\ldots,x_n) = (a_0,\ldots,a_n)$.
\end{claim}
\begin{proof}
 Since $\sum_{i=0}^m x_i\cdot v_i = \sum_{i=0}^m a_i\cdot v_i$,
 we have
 \[\sum_{i=0}^n (x_i-a_i) v_i = \sum_{i=n+1}^m (a_i-x_i)v_i\]
 The zero-intersection condition implies $\sum_{i=0}^n (x_i-a_i) v_i=0 $
 and the linear independence of $v_0,\ldots,v_n$ now implies
 $x_i-a_i=0$ for $i\in [0..n]$.
 
\end{proof}


We proceed to prove knowledge soundness of \bctvprime in the generic group model.
Following \cite{groth16} (see also Section 2 of \cite{BG18} for formal details), it suffices to show knowledge
soundness of \bctvprime as a Non-Interactive Linear Proof (NILP). That is, 
\begin{itemize}
\item The CRS elements and proof elements are the field elements originally encoded in the group elements, rather than the group elements.
\item The malicious prover must output, given only the public input $x\in\F^n$, $\prf = (\pi_A,\pi_B,\pi_C,\pi'_A,\pi'_B,\pi'_C,\pi_H,\pi_K)$ as linear combinations of the CRS elements, such that the verifier equations hold as a polynomial identity in 
$\tau,\rho_A,\rho_B,\alpha_A,\alpha_B,\alpha_C,\gamma,\beta$ as formal variables.
\item To prove knowledge soundness, it then suffices to efficiently derive from the coefficients of these linear combinations a QAP witness for $x$.
\end{itemize}

For \bctvprime, the NILP CRS is now the set of \emph{field elements}

\begin{enumerate}
 \item For $i\in [0..d]$, $\tau^i$.
 \item For $i\in [0..m]$ 
 
 \begin{enumerate}
  \item $\rho_A A_i(\tau)$
\item $\rho_B B_i(\tau)$
\item $\alpha_B\rho_B B_i(\tau)$
\item $\rho_A\rho_B C_i(\tau)$
\item $\alpha_C \rho_A\rho_B C_i(\tau)$ 
\item $\beta(\rho_A A_i(\tau) + \rho_B B_i(\tau) + \rho_A\rho_B C_i(\tau))$ 
\end{enumerate}
\item For $i\in [n+1..m]$, $\alpha_A\rho_A A_i(\tau)$.

\item $\alpha_A,\alpha_B,\alpha_C,\gamma,\beta\gamma,\rho_a\rho_b Z(\tau)$.
\end{enumerate}

and our verification equations are now:
\begin{enumerate}
 \item $\pi'_A= \alpha_A\cdot \pi_A$.
\item $\pi'_B=\alpha_B\cdot \pi_B$.
\item $\pi'_C=\alpha_C\cdot \pi_C$.
\item $\gamma\cdot \pi_K =\beta\gamma\cdot (\PI(x)+\pi_A+\pi_B+\pi_C)$.
\item $(\PI(x)+\pi_A)\cdot \pi_B= \pi_C + \pi_H\cdot Z(\tau)\rho_A\rho_B$,
 \end{enumerate}
where
\[\PI(x)\defeq \rho_A A_0(\tau) + \sum_{i=1}^n x_i\cdot \rho_A A_i(\tau) .\]
 
Suppose now a prover has indeed output a valid proof $\prf = (\pi_A,\pi_B,\pi_C,\pi'_A,\pi'_B,\pi'_C,\pi_H,\pi_K)$
as linear combinations of the CRS elements, such that the verifier equations hold as polynomial identities.
We show how to derive a QAP witness for $x$ from the coefficients of these linear combinations. 
It will be convenient here to think of the proof elements as polynomials only in $\rho_A,\rho_B,\alpha_A,\alpha_B,\alpha_C,\beta,\gamma$ over the ring $\F[\tau]$; i.e. think of the coefficients of these polynomials (not to be confused with the coefficients of the linear combinations of CRS elements) as polynomials in $\tau$.
 

The first equation implies that the linear combination for $\pi'_A$ only has non-zero coefficients for elements with an $\alpha_A$ component, i.e.
\[\pi'_A = \sum_{i=n+1}^m a_i\cdot  \alpha_A\rho_A A_i(\tau) + a^*\cdot \alpha_A \]
for some $a_{n+1},\ldots,a_m,a^* \in \F$.
Which implies 
\[\pi_A = \sum_{i=n+1}^m a_i \cdot \rho_A A_i(\tau) + a^*\]
From the second and third checks, we can similarly conclude:
\[\pi_B = \sum_{i=0}^m b_i\cdot  \rho_B B_i(\tau) + b^*\]

\[\pi_C = \sum_{i=0}^m c_i \cdot \rho_A\rho_B C_i(\tau) + c^*\]
Similarly, the fourth equation tells us that $\pi_K$ must only use with non-zero coefficient elements with a $\beta$ factor,
i.e. the elements $\set{\gamma\beta,\sett{\beta(\rho_A A_i(\tau) + \rho_B B_i(\tau) + \rho_A\rho_B C_i(\tau))}{i\in[0..m]}}$.
Hence we can write
\[\pi_K = \sum_{i=0}^m k_i\cdot\beta(\rho_A A_i(\tau) + \rho_B B_i(\tau) + \rho_A\rho_B C_i(\tau)) + k^*\cdot \gamma\beta\]

and therefore
\begin{equation}\label{eqn:pik}
\PI(x)+\pi_A +\pi_B+\pi_C= \sum_{i=0}^m k_i\cdot(\rho_A A_i(\tau) + \rho_B B_i(\tau) + \rho_A\rho_B C_i(\tau)) + k^*\cdot \gamma.
\end{equation}

Looking at the $\rho_A$ coefficient on both sides of (\ref{eqn:pik}), we have
\[\sum_{i=0}^n x_i \cdot \rho_A A_i(\tau)+\sum_{i=n+1}^m a_i \cdot \rho_A A_i(\tau) = \sum_{i=0}^m k_i\cdot \rho_A A_i(\tau),\]
and so
\[\sum_{i=0}^n x_i \cdot A_i(\tau)+\sum_{i=n+1}^m a_i \cdot A_i(\tau) = \sum_{i=0}^m k_i \cdot A_i(\tau).\]

Invoking Claim \ref{clm:independence}, this implies $k_i=x_i$ for $i\in[0..n]$.
Since the coefficients of $\rho_B,\rho_A\rho_B$ must also match, we have
\[\sum_{i=0}^m b_i\cdot  \rho_B B_i(\tau)= \sum_{i=0}^m k_i \cdot \rho_B B_i(\tau),\sum_{i=0}^m c_i \cdot \rho_A\rho_B C_i(\tau)=\sum_{i=0}^m k_i \cdot\rho_A\rho_B C_i(\tau) \]
So
\[\PI(x)+\pi_A=\sum_{i=0}^m k_i\cdot \rho_A A_i(\tau)+a^*,\pi_B=\sum_{i=0}^m k_i \cdot \rho_B B_i(\tau)+b^*,\pi_C=\sum_{i=0}^m k_i \cdot \rho_A\rho_B C_i(\tau)+c^*\]


Now invoking the fifth equation we have
\begin{equation}\label{eqn:pih}
(\PI(x)+\pi_A)\cdot \pi_B= \pi_C + \pi_H\cdot Z(\tau)\rho_A\rho_B. 
\end{equation}


Denote by $H(\tau)$ the degree at most $d$ polynomial which is the part of $\pi_H$ involving the terms
\sett{\tau_i}{i\in [0..d]} (inspection shows in fact $\pi_H=H(\tau)$, but this is not crucial for us).
The $\rho_A\rho_B$ coefficient of each side in (\ref{eqn:pih}), giving
\[\left(\sum_{i=0}^m k_i \cdot A_i(\tau)\right)\left(\sum_{i=0}^m k_i\cdot  B_i(\tau)\right) = \sum_{i=0}^m k_i\cdot C_i(\tau) + H(\tau)Z(\tau)\]
Since $k_i=x_i$ for $i\in [0..n]$, this means $k_{n+1},\ldots,k_m$ is a QAP witness for public input $x$.

\section*{Acknowledgements}
We thank Sean Bowe for useful discussions.
\bibliographystyle{alpha}
\bibliography{References}

\end{document}
 
